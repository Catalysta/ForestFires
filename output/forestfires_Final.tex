\documentclass[]{article}
\usepackage{lmodern}
\usepackage{amssymb,amsmath}
\usepackage{ifxetex,ifluatex}
\usepackage{fixltx2e} % provides \textsubscript
\ifnum 0\ifxetex 1\fi\ifluatex 1\fi=0 % if pdftex
  \usepackage[T1]{fontenc}
  \usepackage[utf8]{inputenc}
\else % if luatex or xelatex
  \ifxetex
    \usepackage{mathspec}
  \else
    \usepackage{fontspec}
  \fi
  \defaultfontfeatures{Ligatures=TeX,Scale=MatchLowercase}
\fi
% use upquote if available, for straight quotes in verbatim environments
\IfFileExists{upquote.sty}{\usepackage{upquote}}{}
% use microtype if available
\IfFileExists{microtype.sty}{%
\usepackage{microtype}
\UseMicrotypeSet[protrusion]{basicmath} % disable protrusion for tt fonts
}{}
\usepackage[margin=1in]{geometry}
\usepackage{hyperref}
\hypersetup{unicode=true,
            pdftitle={Montesinho Forest Fires},
            pdfauthor={Fernando Anorve-Lopez and Christina Sousa},
            pdfborder={0 0 0},
            breaklinks=true}
\urlstyle{same}  % don't use monospace font for urls
\usepackage{color}
\usepackage{fancyvrb}
\newcommand{\VerbBar}{|}
\newcommand{\VERB}{\Verb[commandchars=\\\{\}]}
\DefineVerbatimEnvironment{Highlighting}{Verbatim}{commandchars=\\\{\}}
% Add ',fontsize=\small' for more characters per line
\usepackage{framed}
\definecolor{shadecolor}{RGB}{248,248,248}
\newenvironment{Shaded}{\begin{snugshade}}{\end{snugshade}}
\newcommand{\KeywordTok}[1]{\textcolor[rgb]{0.13,0.29,0.53}{\textbf{#1}}}
\newcommand{\DataTypeTok}[1]{\textcolor[rgb]{0.13,0.29,0.53}{#1}}
\newcommand{\DecValTok}[1]{\textcolor[rgb]{0.00,0.00,0.81}{#1}}
\newcommand{\BaseNTok}[1]{\textcolor[rgb]{0.00,0.00,0.81}{#1}}
\newcommand{\FloatTok}[1]{\textcolor[rgb]{0.00,0.00,0.81}{#1}}
\newcommand{\ConstantTok}[1]{\textcolor[rgb]{0.00,0.00,0.00}{#1}}
\newcommand{\CharTok}[1]{\textcolor[rgb]{0.31,0.60,0.02}{#1}}
\newcommand{\SpecialCharTok}[1]{\textcolor[rgb]{0.00,0.00,0.00}{#1}}
\newcommand{\StringTok}[1]{\textcolor[rgb]{0.31,0.60,0.02}{#1}}
\newcommand{\VerbatimStringTok}[1]{\textcolor[rgb]{0.31,0.60,0.02}{#1}}
\newcommand{\SpecialStringTok}[1]{\textcolor[rgb]{0.31,0.60,0.02}{#1}}
\newcommand{\ImportTok}[1]{#1}
\newcommand{\CommentTok}[1]{\textcolor[rgb]{0.56,0.35,0.01}{\textit{#1}}}
\newcommand{\DocumentationTok}[1]{\textcolor[rgb]{0.56,0.35,0.01}{\textbf{\textit{#1}}}}
\newcommand{\AnnotationTok}[1]{\textcolor[rgb]{0.56,0.35,0.01}{\textbf{\textit{#1}}}}
\newcommand{\CommentVarTok}[1]{\textcolor[rgb]{0.56,0.35,0.01}{\textbf{\textit{#1}}}}
\newcommand{\OtherTok}[1]{\textcolor[rgb]{0.56,0.35,0.01}{#1}}
\newcommand{\FunctionTok}[1]{\textcolor[rgb]{0.00,0.00,0.00}{#1}}
\newcommand{\VariableTok}[1]{\textcolor[rgb]{0.00,0.00,0.00}{#1}}
\newcommand{\ControlFlowTok}[1]{\textcolor[rgb]{0.13,0.29,0.53}{\textbf{#1}}}
\newcommand{\OperatorTok}[1]{\textcolor[rgb]{0.81,0.36,0.00}{\textbf{#1}}}
\newcommand{\BuiltInTok}[1]{#1}
\newcommand{\ExtensionTok}[1]{#1}
\newcommand{\PreprocessorTok}[1]{\textcolor[rgb]{0.56,0.35,0.01}{\textit{#1}}}
\newcommand{\AttributeTok}[1]{\textcolor[rgb]{0.77,0.63,0.00}{#1}}
\newcommand{\RegionMarkerTok}[1]{#1}
\newcommand{\InformationTok}[1]{\textcolor[rgb]{0.56,0.35,0.01}{\textbf{\textit{#1}}}}
\newcommand{\WarningTok}[1]{\textcolor[rgb]{0.56,0.35,0.01}{\textbf{\textit{#1}}}}
\newcommand{\AlertTok}[1]{\textcolor[rgb]{0.94,0.16,0.16}{#1}}
\newcommand{\ErrorTok}[1]{\textcolor[rgb]{0.64,0.00,0.00}{\textbf{#1}}}
\newcommand{\NormalTok}[1]{#1}
\usepackage{graphicx,grffile}
\makeatletter
\def\maxwidth{\ifdim\Gin@nat@width>\linewidth\linewidth\else\Gin@nat@width\fi}
\def\maxheight{\ifdim\Gin@nat@height>\textheight\textheight\else\Gin@nat@height\fi}
\makeatother
% Scale images if necessary, so that they will not overflow the page
% margins by default, and it is still possible to overwrite the defaults
% using explicit options in \includegraphics[width, height, ...]{}
\setkeys{Gin}{width=\maxwidth,height=\maxheight,keepaspectratio}
\IfFileExists{parskip.sty}{%
\usepackage{parskip}
}{% else
\setlength{\parindent}{0pt}
\setlength{\parskip}{6pt plus 2pt minus 1pt}
}
\setlength{\emergencystretch}{3em}  % prevent overfull lines
\providecommand{\tightlist}{%
  \setlength{\itemsep}{0pt}\setlength{\parskip}{0pt}}
\setcounter{secnumdepth}{0}
% Redefines (sub)paragraphs to behave more like sections
\ifx\paragraph\undefined\else
\let\oldparagraph\paragraph
\renewcommand{\paragraph}[1]{\oldparagraph{#1}\mbox{}}
\fi
\ifx\subparagraph\undefined\else
\let\oldsubparagraph\subparagraph
\renewcommand{\subparagraph}[1]{\oldsubparagraph{#1}\mbox{}}
\fi

%%% Use protect on footnotes to avoid problems with footnotes in titles
\let\rmarkdownfootnote\footnote%
\def\footnote{\protect\rmarkdownfootnote}

%%% Change title format to be more compact
\usepackage{titling}

% Create subtitle command for use in maketitle
\newcommand{\subtitle}[1]{
  \posttitle{
    \begin{center}\large#1\end{center}
    }
}

\setlength{\droptitle}{-2em}

  \title{Montesinho Forest Fires}
    \pretitle{\vspace{\droptitle}\centering\huge}
  \posttitle{\par}
    \author{Fernando Anorve-Lopez and Christina Sousa}
    \preauthor{\centering\large\emph}
  \postauthor{\par}
      \predate{\centering\large\emph}
  \postdate{\par}
    \date{March 12, 2019}


\begin{document}
\maketitle

\section{Introduction}\label{introduction}

The data are from \(n=517\) forest fires occuring in the Montesinho
National Park in northeast Portugal between January 2000 and December
2003. They come to us by way of Paulo Cortez and Anibal Morais at the
University of Minho, in Guimaraes, Portugal. Our goal is to use 12
predictors to model the total burned area of large forest fires within
the park. This could be done using a multiple linear regression model.
Additionally, we would like to identify which values of the predictors
yield fires with burn area less than \(100m^2\), and which ones lead to
larger fires. This could be done using logistic regression.

At each fire occurrence, a fire inspector recorded the date and and time
of the fire, as well as the spatial location of the fire within the park
boundaries. They also recorded the type of vegetation, weather
conditions, total burn area, and six components of the Fire Weather
Index (FWI), a Canadian system for rating fire danger. These included
Fine Fuel Moisture Code (FFMC), Duff Moisture Code (DMC), Drought Code
(DC), Initial Spread Index (ISI), Buildup Index (BUI) and FWI. The FFMC
variable pertains to the moisture content of surface litter; the DMC and
DC variable pertain to the ``moisture content of shallow and deep
organic layers'' (Cortez and Morais 2017); the ISI variable pertains to
fire velocity spread; and finally the BUI variable pertains to the
amount of available fuel. These various codes and scores each contribute
to the final FWI score. Although the individual elements are not
measured on the same scale, higher values indicate greater danger of
fire in all cases.

The researchers did not include BUI and FWI in the data, because these
were strongly collinear with the other predictors.

The variables in the data (Cortez and Morais 2017) are given by:

\begin{enumerate}
\def\labelenumi{\arabic{enumi}.}
\tightlist
\item
  \textbf{X} \emph{(nominal)}: x-axis spatial coordinate within the
  Montesinho park map: 1 to 9
\item
  \textbf{Y} \emph{(nominal)}: y-axis spatial coordinate within the
  Montesinho park map: 2 to 9
\item
  \textbf{month} \emph{(nominal)}: month of the year: ``jan'' to ``dec''
\item
  \textbf{day} \emph{(nominal)}: day of the week: ``mon'' to ``sun''
\item
  \textbf{FFMC} \emph{(ordinal, continuous)}: FFMC index from the FWI
  system: 18.7 to 96.20
\item
  \textbf{DMC} \emph{(ordinal, continuous)}: DMC index from the FWI
  system: 1.1 to 291.3
\item
  \textbf{DC} \emph{(ordinal, continuous)}: DC index from the FWI
  system: 7.9 to 860.6
\item
  \textbf{ISI} \emph{(ordinal, continuous)}: ISI index from the FWI
  system: 0.0 to 56.10
\item
  \textbf{temp} \emph{(interval, continuous)}: temperature in Celsius
  degrees: 2.2 to 33.30
\item
  \textbf{RH} \emph{(ordinal, discrete)}: relative humidity in \%: 15.0
  to 100
\item
  \textbf{wind} \emph{(ratio, continuous)} wind speed in km/h: 0.40 to
  9.40
\item
  \textbf{rain} \emph{(ratio, discrete)}: outside rain in mm/m2 : 0.0 to
  6.4
\item
  \textbf{area} \emph{(ratio, continuous)}: the burned area of the
  forest (in ha): 0.00 to 1090.84
\end{enumerate}

\section{Exploratory Data Analysis}\label{exploratory-data-analysis}

We begin the analysis by reading in the data, plotting it, and viewing
relevant summary statistics.

\begin{Shaded}
\begin{Highlighting}[]
\CommentTok{#data file can be obtained at https://archive.ics.uci.edu/ml/machine-learning-databases/forest-fires/}
\CommentTok{#read data into r}
\NormalTok{forest <-}\KeywordTok{read.csv}\NormalTok{(}\StringTok{"forestfires.csv"}\NormalTok{)}

\CommentTok{#change spatial variables X and Y to factors}
\NormalTok{forest}\OperatorTok{$}\NormalTok{X<-}\KeywordTok{as.factor}\NormalTok{(forest}\OperatorTok{$}\NormalTok{X)}
\NormalTok{forest}\OperatorTok{$}\NormalTok{Y<-}\KeywordTok{as.factor}\NormalTok{(forest}\OperatorTok{$}\NormalTok{Y)}

\CommentTok{#change RH variable to numeric rather than integer (don't think this is needed)}
\CommentTok{#forest$RH<-as.numeric(forest$RH)}

\KeywordTok{head}\NormalTok{(forest)}
\end{Highlighting}
\end{Shaded}

\begin{verbatim}
##   X Y month day FFMC  DMC    DC  ISI temp RH wind rain area
## 1 7 5   mar fri 86.2 26.2  94.3  5.1  8.2 51  6.7  0.0    0
## 2 7 4   oct tue 90.6 35.4 669.1  6.7 18.0 33  0.9  0.0    0
## 3 7 4   oct sat 90.6 43.7 686.9  6.7 14.6 33  1.3  0.0    0
## 4 8 6   mar fri 91.7 33.3  77.5  9.0  8.3 97  4.0  0.2    0
## 5 8 6   mar sun 89.3 51.3 102.2  9.6 11.4 99  1.8  0.0    0
## 6 8 6   aug sun 92.3 85.3 488.0 14.7 22.2 29  5.4  0.0    0
\end{verbatim}

\begin{Shaded}
\begin{Highlighting}[]
\KeywordTok{pairs}\NormalTok{(forest)}
\end{Highlighting}
\end{Shaded}

\includegraphics{forestfires_Final_files/figure-latex/unnamed-chunk-1-1.pdf}

Out of 517 records, there are 247 records with an \texttt{area} value of
0. The \texttt{area} data is considerably right-skewed. This is
explained by Cortez and Morais as a general trend with forest fire data
observed in various locations: either the fires are extinguished quickly
and burn less than \(100m^2\) of area, or they are not easily contained
and end up burning a large area. Fires that burn less than \(100m^2\) of
area are all catalogued as zero. This poses a missing data problem and
prevents us from doing a complete analysis on these types of fires. To
face this problem we can apply the MLR model to the data conditioned to
nonzero burn area only. We remove these data for now, however, we will
make use of them later to compare small fires and large fires using
logistic regression.

\begin{center}\includegraphics{forestfires_Final_files/figure-latex/fig1-1} \end{center}

The values of \texttt{area} also range over several orders of magnitude.
The positive skewness suggests the need of a transformation. Trying log
transformation seems suitable. We log transform \texttt{area} of the
nonzero-valued observations and note that the histogram looks more
symmetric now.

\begin{center}\includegraphics{forestfires_Final_files/figure-latex/fig2-1} \end{center}

We now turn out attention to the other relevant summary statistics of
the data. We notice that certain days and months appear to have more
fires. Also, the distributions of \texttt{FFMC}, \texttt{DMC},
\texttt{DC}, and \texttt{ISI} appear to be somewhat skewed. Finally,
\texttt{rain} does not seem to offer much information for forest fires
with large areas.

\begin{verbatim}
##        X      Y           month     day          FFMC      
##  6      :48   2: 15   aug    :99   fri:43   Min.   :63.50  
##  4      :47   3: 34   sep    :97   mon:39   1st Qu.:90.33  
##  2      :42   4:111   mar    :19   sat:42   Median :91.70  
##  8      :37   5: 68   jul    :18   sun:47   Mean   :91.03  
##  7      :30   6: 38   feb    :10   thu:31   3rd Qu.:92.97  
##  1      :25   8:  1   dec    : 9   tue:36   Max.   :96.20  
##  (Other):41   9:  3   (Other):18   wed:32                  
##       DMC              DC             ISI              temp      
##  Min.   :  3.2   Min.   : 15.3   Min.   : 0.800   Min.   : 2.20  
##  1st Qu.: 82.9   1st Qu.:486.5   1st Qu.: 6.800   1st Qu.:16.12  
##  Median :111.7   Median :665.6   Median : 8.400   Median :20.10  
##  Mean   :114.7   Mean   :570.9   Mean   : 9.177   Mean   :19.31  
##  3rd Qu.:141.3   3rd Qu.:721.3   3rd Qu.:11.375   3rd Qu.:23.40  
##  Max.   :291.3   Max.   :860.6   Max.   :22.700   Max.   :33.30  
##                                                                  
##        RH             wind            rain             Larea        
##  Min.   :15.00   Min.   :0.400   Min.   :0.00000   Min.   :-2.4079  
##  1st Qu.:33.00   1st Qu.:2.700   1st Qu.:0.00000   1st Qu.: 0.7608  
##  Median :41.00   Median :4.000   Median :0.00000   Median : 1.8516  
##  Mean   :43.73   Mean   :4.113   Mean   :0.02889   Mean   : 1.8448  
##  3rd Qu.:53.00   3rd Qu.:4.900   3rd Qu.:0.00000   3rd Qu.: 2.7358  
##  Max.   :96.00   Max.   :9.400   Max.   :6.40000   Max.   : 6.9947  
## 
\end{verbatim}

As categorical data, the \texttt{month} variable can be divided into
subsets defined by seasons.

\includegraphics{forestfires_Final_files/figure-latex/unnamed-chunk-3-1.pdf}

We can also redefine Jan, Feb, Mar as Winter; Apr, May, Jun as Spring;
Jul, Aug, Sep as Summer, and Oct, Nov, Dec, as Fall, because seasonal
weather conditions can have an effect on incidence of wildfires. In this
case, Summer records show more variability than any other season.

\begin{center}\includegraphics{forestfires_Final_files/figure-latex/fig3-1} \end{center}

The \texttt{day} variable can be divided into subsets ``weekdays'' and
``weekends'', since forest fires could be explained by exposure to human
beings who visit the park on the weekends.

\begin{center}\includegraphics{forestfires_Final_files/figure-latex/fig 4-1} \end{center}

\begin{center}\includegraphics{forestfires_Final_files/figure-latex/fig 4-2} \end{center}

With respect to the spacial coordinates, it seems that the \texttt{X}
and \texttt{Y} coordinates would be better explained by assigning each
grid of the map its own category, to see if one area of the map appears
more than others. Indeed, there are 4 regions (out of 81) of the map
that had 20 or more fires in this 3-year period. It may be interesting
to compare the sizes of fires in these areas, and whether these fires
are bigger than fires in other areas of the park.

\begin{center}\includegraphics{forestfires_Final_files/figure-latex/unnamed-chunk-5-1} \end{center}

We also note that, after removing the forest fires with zero area, there
are only 2 records with nonzero values for \texttt{rain}. Hence it does
not make sense to include this predictor in our current analysis, and we
remove it. We may take another look at this later in the logistic
regression setting. We now view the continuous predictors after removal
of this variable:

\includegraphics{forestfires_Final_files/figure-latex/unnamed-chunk-6-1.pdf}

For the fire index variables, we note that the quantiles appear to be a
bit skewed, particularly for \texttt{FFMC} and \texttt{DMC}, so we
suspect that transformation of these variables may be in order. The
boxplots of the fire index variables appear below.

\begin{center}\includegraphics{forestfires_Final_files/figure-latex/unnamed-chunk-7-1} \end{center}

Noticing that \texttt{FFMC} is negatively skewed, we can try to
normalize it by reflecting it, adding an appropriate value to make it
positive, and finally applying a log transformation.
\[f(\text{FFMC}) = \log(-\text{FFMC}+ \max(\text{FFMC})+1) \]

This seems to correct the skew.

\begin{center}\includegraphics{forestfires_Final_files/figure-latex/unnamed-chunk-8-1} \end{center}

The final cleaned continuous variables appear below.

\includegraphics{forestfires_Final_files/figure-latex/unnamed-chunk-9-1.pdf}

Finally, we comment on collinearity in this data. It appears that
\texttt{DMC} and \texttt{DC} are linearly correlated. There may also be
correlation between \texttt{temp} and \texttt{ISI} and \texttt{RH} and
\texttt{temp}. In fact, \texttt{temp} appears to be linearly correlated
with just about all the other predictors. There is also an interesting
(and perhaps troubling!) curvilinear relationship between \texttt{DMC}
and \texttt{ISI} as well as \texttt{DMC} and \texttt{temp}. To
summarize, there appears quite a bit of multicollinearity in this data.
This may be because the fire indexes are based on the various weather
variables, and because weather variables such as temperature and
humidity are related to one another.

\section{Summary}\label{summary}

At this point there do not appear to be any continuous variables that
look like strong predictors for \texttt{Larea}. Perhaps the discrete
variables will become more important as we attempt to model this
response. We also think that a logistic regression approach might be
more productive in determining which variables are significant to
predict whether the burn area of a wildfire will be larger or smaller
than \(100m^2.\)

We could also attempt to interpolate the missing data for the zero-area
entries by randomly sampling areas between \(0\) and \(100m^2\)
according to a distribution consistent with the rest of the data, to see
if this improves our results.

To model \texttt{Larea}, we also might proceed by seeking out suitable
transformations of these variables by applying the Yeo-Johnson method.
The \texttt{car} package function \texttt{powerTransform} suggests that
the cleaned data could be normalized with a few reasonable
transformations.

\begin{Shaded}
\begin{Highlighting}[]
\KeywordTok{require}\NormalTok{(car)}
\KeywordTok{summary}\NormalTok{(}\KeywordTok{powerTransform}\NormalTok{(}\KeywordTok{cbind}\NormalTok{(TrFFMC,DMC,DC,ISI,temp,RH,wind)}\OperatorTok{~}\DecValTok{1}\NormalTok{,forest5,}\DataTypeTok{family=}\StringTok{"yjPower"}\NormalTok{))}
\end{Highlighting}
\end{Shaded}

\begin{verbatim}
## yjPower Transformations to Multinormality 
##        Est Power Rounded Pwr Wald Lwr Bnd Wald Upr Bnd
## TrFFMC    2.0905        2.00       1.8008       2.3802
## DMC       0.3221        0.33       0.2270       0.4171
## DC        1.3926        1.39       1.1822       1.6029
## ISI       0.0516        0.00      -0.1214       0.2247
## temp      1.1356        1.00       0.8984       1.3729
## RH        0.1952        0.00      -0.1064       0.4968
## wind      0.3811        0.50       0.1109       0.6514
## 
##  Likelihood ratio test that all transformation parameters are equal to 0
##                                        LRT df       pval
## LR test, lambda = (0 0 0 0 0 0 0) 643.3795  7 < 2.22e-16
\end{verbatim}

Also, a regression tree suggests that \texttt{RH}, \texttt{DC},
\texttt{ISI}, and the transformed \texttt{FFMC} variables may still be
significant in predicting \texttt{Larea}. These may be worth adding to
the MLR model one-at-a-time to see if a desirable level of \(R^2\) can
be achieved (see figure on next page).

\begin{Shaded}
\begin{Highlighting}[]
\KeywordTok{require}\NormalTok{(tree)}
\NormalTok{data.tree <-}\StringTok{ }\KeywordTok{tree}\NormalTok{(Larea}\OperatorTok{~}\NormalTok{., }\DataTypeTok{data=}\NormalTok{forest6[,}\DecValTok{6}\OperatorTok{:}\DecValTok{13}\NormalTok{],}\DataTypeTok{mincut=}\DecValTok{15}\NormalTok{, }\DataTypeTok{minsize=}\DecValTok{35}\NormalTok{)}
\KeywordTok{plot}\NormalTok{(data.tree, }\DataTypeTok{type=}\StringTok{"uniform"}\NormalTok{)}
\KeywordTok{text}\NormalTok{(data.tree)}
\end{Highlighting}
\end{Shaded}

\begin{center}\includegraphics{forestfires_Final_files/figure-latex/fig6-1} \end{center}

\section*{Citations}\label{citations}
\addcontentsline{toc}{section}{Citations}

\hypertarget{refs}{}
\hypertarget{ref-cortezff}{}
Cortez, Paulo, and Anibal Morais. 2017. ``A Data Mining Approach to
Predict Forest Fires Using Meteorological Data.'' Edited by J Neves, M F
Santos, and J Machado. \emph{New Trends in Artificial Intelligence,
Proceedings of the 13th EPIA 2007 - Portuguese Conference on Artificial
Intelligence}, December. Guimaraes, Portugal, 512--23.
\url{http://www.dsi.uminho.pt/~pcortez/fires.pdf}.


\end{document}
